\documentclass[luiss,english]{academicthesis}

\usepackage{lipsum} %Dummy package

% ------------------------------------------------
% Metadata (demonstrates your API)
% ------------------------------------------------
\thtitle{A Demonstration of the \texttt{academicthesis} Class}
\thsubtitle{Design, Structure, and Extensibility}

\candidate{John Doe}
\degree{Bachelor of Science in Computer Science}
\chair{Economics and Data Policy}
\academicyear{2025--2026}
\idnumber{123456}

\supervisor{Prof.\ Alice Smith}
\cosupervisor{Prof.\ Bob Johnson}

\copyyear{2026}
\authoremail{john.doe@example.com}

% Gender example
\gender{male}

% ------------------------------------------------
\begin{document}

% =================================================
% FRONT MATTER
% =================================================
\frontmatter

\frontpage

% ---------------- Dedication ---------------------
\dedication{%
To all students who choose \LaTeX{} not because it is easy,  
but because it is correct.
}

% ---------------- Abstract -----------------------
\begin{abstractthesis}
This document provides a complete, copyright-free example of a thesis
produced using the \texttt{thesis} class.
It demonstrates the structure, layout, and extensibility of the class,
including front matter, chapter organization, and typographic conventions.
All placeholder text is generated using public-domain material.
\end{abstractthesis}


% ---------------- Table of Contents --------------
\tableofcontents

% =================================================
% MAIN MATTER
% =================================================
\mainmatter

% =================================================
\chapter{Introduction}
\section{Background}

\lipsum[1-6]

\section{Motivation}
\lipsum[7-12]

\subsection{Academic Writing with \LaTeX}
\lipsum[13-15]

% =================================================
\chapter{Design of the Thesis Class}
\section{Class Architecture}
\lipsum[16-21]

\section{Metadata Management}
\lipsum[22-26]

\subsection{Validation and Error Handling}
\lipsum[27-29]

% =================================================
\chapter{Implementation Details}
\section{Key--Value Interfaces}
\lipsum[30-35]

\section{University-Specific Layouts}
\lipsum[36-40]

\subsection{Extensibility Considerations}
\lipsum[41-44]

% =================================================
\chapter{Results and Discussion}
\section{Visual Consistency}
\lipsum[45-49]

\section{Usability for Students}
\lipsum[50-54]

\subsection{Comparison with Standard Classes}
\lipsum[55-57]

% =================================================
\chapter{Conclusion}
\section{Summary}
\lipsum[58-60]

\section{Future Work}
\lipsum[61-63]

% =================================================
% BACK MATTER
% =================================================
\backmatter

\chapter*{Final Remarks}
\addcontentsline{toc}{chapter}{Final Remarks}
\lipsum[64-66]


% ---------------- Acknowledgments ----------------
\begin{acknowledgements}
The author would like to thank the open-source \LaTeX{} community for
providing robust tools for academic writing, and all educators who
encourage clarity, precision, and reproducibility in scholarly work.
\end{acknowledgements}


\end{document}
